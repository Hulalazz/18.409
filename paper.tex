\documentclass[11pt]{article}

\usepackage{amsthm, amssymb, amsmath}
\usepackage{rotating}
\usepackage{verbatim}
\usepackage{enumerate}
\usepackage[hmarginratio=1:1,top=32mm,columnsep=20pt]{geometry}
\usepackage{multicol} % Used for the two-column layout of the document
\usepackage[hang, small,labelfont=bf,up,textfont=it,up]{caption}
\usepackage{booktabs} % Horizontal rules in tables
\usepackage{hyperref} % For hyperlinks in the PDF
\usepackage[shortlabels]{enumitem}
\usepackage{abstract} % Allows abstract customization
\renewcommand{\abstractnamefont}{\normalfont\bfseries} % Set the ``Abstract'' text to bold
\renewcommand{\abstracttextfont}{\normalfont\small\itshape} % Set the abstract itself to small italic text

\usepackage{titlesec} % Allows customization of titles
\titlelabel{\thetitle.\quad}
%\renewcommand\thesection{\Roman{section}} % Roman numerals for the sections
%\renewcommand\thesubsection{\Roman{subsection}} % Roman numerals for subsections
%\titleformat{\section}[block]{\large\scshape\centering}{\thesection.}{1em}{} % Change the look of the section titles
%\titleformat{\subsection}[block]{\large}{\thesubsection.}{1em}{} % Change the look of the section titles

\usepackage{fancyhdr} % Headers and footers
\pagestyle{fancy} % All pages have headers and footers
\fancyhead{} % Blank out the default header
\fancyfoot{} % Blank out the default footer
\fancyhead[C]{May 2015} % Custom header text
\fancyfoot[RO,LE]{\thepage} % Custom footer text

\linespread{1.05}
\bibliographystyle{plain}
\newtheorem{lemma}{Lemma}
\newtheorem{corollary}{Corollary}
\theoremstyle{definition}
\newtheorem{theorem}{Theorem}


% --------------------
% TITLE SECTION
% --------------------

\title{\vspace{-5mm}\fontsize{22pt}{10pt}\selectfont Smoothed Analysis in Learning: Tensors and $k$-Means}

\author{
\large
\textsc{Hunter Lang} \\
\normalsize MIT \\
\normalsize \href{mailto:hjl@mit.edu}{\texttt{hjl@mit.edu}}\\
\and
\textsc{Carlos Cortez} \\
\normalsize MIT \\
\normalsize \href{mailto:cortezc@mit.edu}{\texttt{cortezc@mit.edu}}\\
\vspace{-5mm}
}
\date{}

\begin{document}
\maketitle

%--------------------
% ABSTRACT
%--------------------

\begin{abstract}
\noindent Most learning problems are hard in the worst-case, so much
of the current research focuses on finding good heuristics,
polynomial-time approximation algorithms, or special cases with
provable accuracy and runtime guarantees. But many algorithms that are
inefficient in the worst case consistently seem to run fast in
practice (Simplex being the classical example). Smoothed analysis
gives a theoretical framework for understanding performance on
real-world data, specifically for problems where some component is not
adversarial. This is a natural assumption in learning, since data in
the learning setting are usually prone to measurement or modeling
noise. We survey the (very much ongoing) application of smoothed
analysis to learning problems by way of two examples: $k$-means and
tensor decomposition.
\end{abstract}

% todo: add formal definition of SA found in \cite{SAtwo} to this section.

\section{Introduction}
Spielman and Teng \cite{SA}, \cite{SAtwo} introduced smoothed analysis
to give a more suitable framework for predicting real-life algorithm
performance. Not every algorithm that runs fast in practice is
polynomial-time; worst-case analysis falls short of explaining why
some ``slow'' algorithms are empirically quite efficient. As a first
application of their techniques, the original paper \cite{SA} gave a
proof that the Simplex algorithm has polynomial ``smoothed
complexity'': if you assume the input data are subject to random
noise, Simplex runs in expected polynomial time. This sparked a host
of papers applying smoothed analysis to classic combinatorial
optimization problems. [EXAMPLES]. The key assumption of the smoothed
setting is that some component of the problem is not adversarial. The
hope is that worst-case instances are somehow isolated in the input
space (indeed, the worst-case inputs to many algorithms are intricate
and fragile), so that any real data is unlikely to be a worst-case
instance. In recent years, there has been an increasing trend of
applying smoothed analysis to learning problems \cite{SAtwo},
\cite{SAkmeans}, \cite{PAC}, \cite{TD}. (FIX): There are two active fronts:
some researchers have followed the standard line of smoothed analysis,
focusing on algorithm \emph{runtimes}. Others have aimed to show that
algorithms still \emph{work} on perturbed instances of problems; this
has also been called smoothed analysis, as in \cite{TD}. We give
examples of both. [explain what 4 and 6 are] In the next section, we
give a brief formal definition of smoothed analysis, following
\cite{SAtwo}, to set the environment for the examples that follow. In
section 3 we discuss at a high level the result of \cite{SAkmeans},
showing that the $k$-means method has polynomial smoothed runtime. In
section 4 we examine a tensor decomposition algorithm in a smoothed
setting, showing not only that it runs in polynomial time on
smoothed instances, but also that it works provably well, i.e. that it
recovers good factors. In section 5 we conclude and offer some
speculation about directions for further research.

\section{Smoothed Analysis: Preliminaries}
When $A$ is an algorithm for solving problem $P$, we denote by
$T_A[x]$ the runtime of $A$ on an input instance $x$. We consider the
input domain $\Omega$ to problem $P$ as the union of disjoint
subdomains $\{\Omega_1,\ldots, \Omega_n,\ldots\}$, where $\Omega_n$ is
the family of all inputs of size $n$. Given this setting we can define
worst-case and smoothed runtimes. The \emph{worst-case measure} is defined as

$$\mbox{WC}_A(n) = \max_{x \in \Omega_n}T_A[x].$$
The \emph{smoothed complexity} of $A$ with $\sigma$-Gaussian perturbations is given by
$$\mbox{Smoothed}_A^{\sigma}(n) = \max_{x\in [-1,1]^n}
\mathbb{E}_g[T_A(x + g)],$$ where $g$ is a Gaussian random vector of
variance $\sigma^2$. The original input $x$ is perturbed by $g$ to
obtain a new input $x + g$, which is given to the algorithm.

We can view $\sigma$ as a parameter that interpolates between
worst-case and average-case analysis. When $\sigma = 0$ the smoothed
complexity is the same as the worst-case measure. When $\sigma$ is
very large, the smoothed complexity approaches the average-case
complexity, since $g$ dominates the original $x$. Here we are
interested in the case where $\sigma$ is small compared to $||x||$, so
that $x + g$ is only a slight perturbation. We now give an example of
using smoothed analysis to analyze the runtime of a learning
algorithm, $k$-means.

\section{$k$-means}
The $k$-means algorithm solves the \emph{clustering problem}, set as
follows. Given a set $S$ of points in a (generally high-dimensional)
space $\mathbb{R}^d$, output a $k$-partition of $S$ (a collection of
$k$ ``clusters'') such that the elements in each cluster are similar
under a chosen metric. $k$-means begins by arbitrarily choosing the
$k$ cluster centers (how to choose these centers is the subject of
much research and debate). The main algorithm consists of two
iterative steps: first, it assigns each point in $S$ to the closest
center. Next, it computes the center of mass of each existing cluster
(the cluster mean). These means become the new cluster centers and the
assignment process repeats. The algorithm terminates when the local
improvements to cluster centers no longer affect the assignment of
points.

In \cite{KMworstcase}, Vattani shows that an adversary can choose
initial centers that will force $k$-means to iterate $2^{\Omega(n)}$
times for $d \ge 2$. But $k$-means is usually very fast in practice
(see e.g. \cite{kmeansfast}). So the setting seems ripe for smoothed
analysis: there is a large gap between theoretical and observed
runtimes, and the assumption that data points are subject to noise is
reasonable. Before we proceed it is important to point out that aside
from a slow runtime, there is another problem with $k$-means: there
are no guarantees on the quality of the solution it produces; the
algorithm is only guaranteed to converge to a local optimum and is
sensitive to the initial choice of cluster centers
\cite{kmeanssensitive}. There are mild conditions under which a
variant of the above algorithm outputs a provably good clustering
\cite{kmeansprovable}, but these techniques do not use what we
consider smoothed analysis, instead defining ``separability
conditions'' on the input instance. We focus on the first problem, a
problem of runtime, and summarize the argument of Arthur et
al. \cite{SAkmeans}, the main result of which is the following
theorem.

\begin{theorem}
Fix a set $S^{\prime} \subset [0,1]^d$ of $n$ data points, and
independently perturb each point in $S^{\prime}$ by a Gaussian
distribution with mean 0 and variance $\sigma^2$. Label the new set
$S$. Then the expected running time of $k$-means on $S$ is bounded by
a polynomial in $n$ and $1/\sigma$.
\end{theorem}
Again, note that this theorem gives no guarantees on the optimality of
the output. The argument essentially 

\section{Tensor Decomposition}
(FIX): Among other problems, tensor decomposition methods can be used
to learn topic models, multi-view mixture models, phylogenetic trees,
and detect communities. Tensors are especially useful in learning
because under mild conditions, the factors of a rank-$k$ 3-tensor
decomposition $T = \sum_{i=1}^ka_i \otimes b_i \otimes c_i$ are unique
\cite{tensorunique}. These conditions extend naturally to higher-order
tensors. Assume we are given a $l$-order tensor $T$ of dimension $n$
(so $T\in \mathbb{R}^{n^{\times l}}$) and of rank $R$ and we want to
decompose it into a sum of rank-one tensors, i.e.:
$$T = \sum_{i=1}^R u_i^{(1)}\otimes u_i^{(2)}\otimes\dots\otimes
u_i^{(l)}$$ We first describe conditions under which such
decomposition is unique and computable in polynomial time.

The algorithm above works well in the \emph{exact} case, when we know
$T$. But in the learning setting, we don't know $T$ exactly--we have
an approximation $\~ T$ of $T$, since we can only take polynomially
many samples [VAGUE?]. Hence, we need to make sure the algorithm is
robust in the presence of noise. In particular, we prove that if each
entry of $E=T-\~ T$ is bounded by $\epsilon \cdot
\mbox{poly}_l(1/k,1/n,1/\delta)$, then we can compute a decomposition
in which each rank-$1$ tensor in the decomposition has an additive
error of at most $\epsilon$.

\section{Conclusion}

\begin{thebibliography}{1}
  \bibitem{SA}
    Daniel A. Spielman and Shang{-}Hua Teng.
    ``Smoothed Analysis of Algorithms: Why the Simplex Algorithm Usually takes Polynomial Time.''
    \emph{Journal of the ACM, Vol 51 (3)},
    pp. 385 - 463,
    2004.

  \bibitem{SAtwo}
    Daniel A. Spielman and Shang{-}Hua Teng.
    ``Smoothed Analysis: An Attempt to Explain the Behavior of Algorithms in Practice''
    \emph{Communications of the ACM, Vol. 52 No. 10},
    pp. 76-84,
    2009.

  \bibitem{KMworstcase}
    Andrea Vattani.
    ``k-means Requires Exponentially Many Iterations Even in the Plane.'' 
    \emph{ Proc. of the 25th ACM Symp. on Computational Geometry (SoCG)}, 
    pp 324–332, 
    2009.
    
  \bibitem{SAkmeans}
    David Arthur, Bodo Manthey, and Heiko Roeglin.
    ``Smoothed Analysis of the $k$-means Method.''
    2010.

  \bibitem{PAC}
    Adam Tauman Kalai, Alex Samorodnitsky, and Shang{-}Hua Teng.
    ``Learning and Smoothed Analysis'',
    \emph{IEEE 54th Annual Symposium on Foundations of Computer Science}, 
    pp. 395-404,
    2009.

  \bibitem{TD}
    Aditya Bhaskara, Moses Charikar, Ankur Moitra, and Aravindan Vijayaraghavan.
    ``Smoothed Analysis of Tensor Decomposition'',
    \emph{CoRR},
    2013.

  \bibitem{tensorunique}
    J. Kruskal. 
    ``Three-way Arrays: Rank and Uniqueness of Trilinear Decompositions'',
    \emph{Linear Algebra and Applications}, 
    18:95–138, 
    1977.
  \bibitem{kmeansprovable}
    Rafail Ostrovsky and Yuval Rabani and et al.
    ``The Effectiveness of Lloyd-Type Methods for the k-Means Problem'',

  \bibitem{kmeanssensitive}
    G. Milligan. 
    ``An examination of the effect of six types of error perturbation on fifteen clustering
    algorithms,'' 
    \emph{Psychometrika}, 
    45:325–342, 1980.

\end{thebibliography}
\end{document}
