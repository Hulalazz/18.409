\documentclass[11pt]{article}

\usepackage{amsthm, amssymb, amsmath}
\usepackage{rotating}
\usepackage{verbatim}
\usepackage{enumerate}
\usepackage[hmarginratio=1:1,top=32mm,columnsep=20pt]{geometry}
\usepackage{multicol} % Used for the two-column layout of the document
\usepackage[hang, small,labelfont=bf,up,textfont=it,up]{caption}
\usepackage{booktabs} % Horizontal rules in tables
\usepackage{hyperref} % For hyperlinks in the PDF
\usepackage[shortlabels]{enumitem}
\usepackage{abstract} % Allows abstract customization
\renewcommand{\abstractnamefont}{\normalfont\bfseries} % Set the ``Abstract'' text to bold
\renewcommand{\abstracttextfont}{\normalfont\small\itshape} % Set the abstract itself to small italic text

\usepackage{titlesec} % Allows customization of titles
\titlelabel{\thetitle.\quad}
%\renewcommand\thesection{\Roman{section}} % Roman numerals for the sections
%\renewcommand\thesubsection{\Roman{subsection}} % Roman numerals for subsections
%\titleformat{\section}[block]{\large\scshape\centering}{\thesection.}{1em}{} % Change the look of the section titles
%\titleformat{\subsection}[block]{\large}{\thesubsection.}{1em}{} % Change the look of the section titles

\usepackage{fancyhdr} % Headers and footers
\pagestyle{fancy} % All pages have headers and footers
\fancyhead{} % Blank out the default header
\fancyfoot{} % Blank out the default footer
\fancyhead[C]{May 2015} % Custom header text
\fancyfoot[RO,LE]{\thepage} % Custom footer text

\linespread{1.05}
\bibliographystyle{plain}
\newtheorem{lemma}{Lemma}
\newtheorem{corollary}{Corollary}
\theoremstyle{definition}
\newtheorem{theorem}{Theorem}


% --------------------
% TITLE SECTION
% --------------------

\title{\vspace{-5mm}\fontsize{22pt}{10pt}\selectfont Smoothed Analysis in Learning: Tensors and $k$-Means}

\author{
\large
\textsc{Hunter Lang} \\
\normalsize MIT \\
\normalsize \href{mailto:hjl@mit.edu}{\texttt{hjl@mit.edu}}\\
\and
\textsc{Carlos Cortez} \\
\normalsize MIT \\
\normalsize \href{mailto:cortezc@mit.edu}{\texttt{cortezc@mit.edu}}\\
\vspace{-5mm}
}
\date{}

\begin{document}
\maketitle

%--------------------
% ABSTRACT
%--------------------

\begin{abstract}
\noindent Most learning problems are hard in the worst-case, so much
of the current research focuses on finding good heuristics,
polynomial-time approximation algorithms, or special cases with
provable accuracy and runtime guarantees. But many algorithms that are
inefficient in the worst case consistently seem to run fast in
practice (Simplex being the classical example). Smoothed analysis
gives a framework for better understanding performance on real-world
data, specifically for problems where some component is not
adversarial. This is a natural assumption in learning, since data in
the learning setting are usually prone to measurement or modeling
noise. We survey the (very much ongoing) application of smoothed
analysis to learning problems by way of two examples: $k$-means and
tensor decomposition.
\end{abstract}

\section{Introduction}
Spielman and Teng \cite{SA}, \cite{SAtwo} introduced smoothed analysis
to give a more suitable framework for predicting real-life algorithm
performance. Not every algorithm that runs fast in practice is
polynomial-time; worst-case analysis falls short of explaining why
some ``slow'' algorithms are empirically quite efficient. As a first
application of their techniques, the original paper \cite{SA} gave a
proof that the Simplex algorithm has polynomial ``smoothed
complexity'': that is, if you assume the data are subject to random
noice (the kind that would arise in many practical scenarios), Simplex
runs in expected polynomial time. This sparked a host of papers
applying smoothed analysis to classic combinatorial optimization
problems. [EXAMPLES]. The key assumption of the smoothed analysis
setting is that some component of the problem is not adversarial. The
hope is that worst-case instances are somehow isolated in the input
space (indeed, the worst-case inputs to many algorithms are intricate
and fragile), so that any real data is unlikely to be a worst-case
instance. In recent years, there has been an increasing trend of
applying smoothed analysis to learning problems \cite{SAtwo},
\cite{SAkmeans}, \cite{PAC}, \cite{TD}. Some researchers have followed
the standard line of smoothed analysis, focusing on algorithm
\emph{runtimes}. Others have expanded the ideas of smoothed analysis
to show that algorithms still work under noise. We will explain the
difference by an example.
\section{Tensor Decomposition}
Among other problems, tensor decomposition methods can be used to
learn topic models, multi-view mixture models, phylogenetic trees, and
detect communities [CITE--course monograph?]. In practice, tensors we
want to decompose into factors are random.

\section{$k$-means}
$k$-means is an algorithm 
\section{Conclusion}

\begin{thebibliography}{1}
  \bibitem{SA}
    Daniel A. Spielman and Shang{-}Hua Teng,
    ``Smoothed Analysis of Algorithms: Why the Simplex Algorithm Usually takes Polynomial Time.''
    \emph{Journal of the ACM, Vol 51 (3)},
    pp. 385 - 463,
    2004.

  \bibitem{SAtwo}
    Daniel A. Spielman and Shang{-}Hua Teng,
    ``Smoothed Analysis: An Attempt to Explain the Behavior of Algorithms in Practice''
    \emph{Communications of the ACM, Vol. 52 No. 10},
    pp. 76-84,
    2009.

  \bibitem{KMworstcase}
    Andrea Vattani,
    ``k-means Requires Exponentially Many Iterations Even in the Plane.'' 
    \emph{ Proc. of the 25th ACM Symp. on Computational Geometry (SoCG)}, 
    pp 324–332, 
    2009.
    
  \bibitem{SAkmeans}
    David Arthur, Bodo Manthey, and Heiko Roeglin,
    ``Smoothed Analysis of the $k$-means Method.''
    2010.

  \bibitem{PAC}
    Adam Tauman Kalai, Alex Samorodnitsky, and Shang{-}Hua Teng, 
    ``Learning and Smoothed Analysis'',
    \emph{IEEE 54th Annual Symposium on Foundations of Computer Science}, 
    pp. 395-404,
    2009.

  \bibitem{TD}
    Aditya Bhaskara, Moses Charikar, Ankur Moitra, and Aravindan Vijayaraghavan,
    ``Smoothed Analysis of Tensor Decomposition'',
    CoRR,
    2013.

\end{thebibliography}
\end{document}
